\section{Тема 17}

\subsection*{Булеви функции (на една и две променливи). Съществени и несъществени променливи.}

\begin{note}
    \mexpr{J_2 = \{0, 1\}, J_2^n = \underbrace{J_2 \times J_2 \times ... \times J_2}_\text{\(n\) множителя}}
\end{note}

\begin{definition}
    \underline{Булева ф-я} на \(n\) променливи е всяка ф-я \mexpr{f: J_2^n \to J_2} за някое \(n \ge 1\).
\end{definition}

Булевите ф-ии на 0 променливи се отъждествяват с булевите константи 0 и 1, защото 0-кратното декартово 
произведение е {()}, откъдето домейнът е едноелементен и има точно две булеви ф-ии на 0 променливи.

\begin{definition}
    Нека \mexpr{f = (x_1, ..., x_n)} е булева ф-я. Променливата \(x_i\) се нарича \underline{фиктивна (
    несъществена)}, ако
    \begin{equation*}
        f(x_1, x_2, ..., x_{i - 1}, 0, x_{i + 1}, ..., x_n) = f(x_1, ..., x_{i - 1}, 1, x_{i + 1}, ..., x_n)
    \end{equation*}
    за всяка стойност на \((n - 1)\)-вектора \mexpr{x_1x_2...x_{i - 1}x_{i + 1}...x_n}.
    Променлива, която не е фиктивна, се нарича \underline{съществена}.
\end{definition}

\subsection*{Формула над множество булеви функции. Булева функция, съответна на дадена формула.}



\subsection*{Свойства на функциите на една и две променливи.}

\begin{enumerate}
    \item Комутативност
    \begin{align*}
        xy &= yx \\
        x \vee y &= y \vee x \\
        x \oplus y &= y \oplus x
    \end{align*}
    \item Асоциативност
    \begin{align*}
        (xy)z &= x(yz) \\
        (x \vee y) \vee z &= x \vee (y \vee z) \\
        (x \oplus y) \oplus z &= x \oplus (y \oplus z)
    \end{align*}
    \item Идемпотентност
    \begin{align*}
        xx &= x \\
        x \oplus x &= \tilde{0} \\
        x \vee x &= x
    \end{align*}
    \item Свойство на отрицанието
    \begin{align*}
        x\overline{x} &= \tilde{0} \\
        x \vee \overline{x} &= \tilde{1} \\
        x \oplus \overline{x} &= \tilde{1} \\
        \overline{\overline{x}} &= x
    \end{align*}
    \item Свойство на константите
    \begin{align*}
        x\tilde{0} &= \tilde{0} \\
        x\tilde{1} &= \tilde{1} \\
        x \vee \tilde{1} &= \tilde{1} \\
        x \oplus \tilde{1} &= \overline{x} \\
        x \oplus \tilde{0} &= x \\
        x \vee \tilde{0} &= x
    \end{align*}
    \item Закони на де Морган
    \begin{align*}
        \overline{x \vee y} &= \overline{x} \land \overline{y} \\
        \overline{x \land y} &= \overline{x} \vee \overline{y}
    \end{align*}
\end{enumerate}

\subsection*{Допълнителни бележки}
М-вото от всички булеви ф-ии на \(n\) променливи е \(F_2^n\). \\
М-вото от всички булеви ф-ии е \mexpr{F_2 = \cup_{n \in \mathbb{N}} F_2^n}. \\
Елемените на \(J_2^n\) са булевите вектори с дължина \(n\). Записваме ги като \mexpr{a = (a_1, ..., a_n)} 
или \mexpr{a = a_1a_2...a_n}. \\
Нека \mexpr{x, y \in J_2, x = \overline{y}} \totw 