\section{Тема 9}

% Основни комбинаторни конфигурации. Формули за броя на елементите на
% основните комбинаторни конфигурации – наредени и ненаредени, с
% повтаряне и без повтаряне.
\subsection*{Основни комбинаторни конфигурации и формули за броя на елементите им} 

\begin{definition}[Конфигурации с наредба и повтаряне]
    Множеството от конфигурациите с наредба и повтаряне с големина \(m\) (т.е. броят на елементите в нея) 
    над опорно множество \(X\) с мощност \(n\) означаваме с "\(K_\text{н, п}(n, m)\)". Елементите му са 
    наредените \(m\)-торки (вектори с дължина \(m\)), чиито елементи са от опорното множество \(X\). 
    Тогава \(K_\text{н, п}(n, m) = X^m\). \\
    От обобщения принцип на умножението: \mexpr{|K_\text{н, п}(n, m)| = |X^m| = n^m}
\end{definition}

\begin{example}
    Нека \mexpr{X = \{a, b, c\}, m = 2}. \\
    Тогава \mexpr{K_\text{н, п}(3, 2) = \{(a, a); (a, b); (a; c); (b, a); (b, b); (b, c); (c, a); 
    (c, b); (c, c) \}} и \mexpr{|K_\text{н, п}(3, 2)| = 3^2 = 9}.
\end{example}

\begin{definition}[Конфигурации с наредба без повтаряне]
    Множеството от конфигурациите с наредба без повтаряне с големина \(m\) 
    над опорно множество \(X\) с мощност \(n\) означаваме с "\(K_\text{н}(n, m)\)". Елементите му са 
    наредените \(m\)-торки без повтаряне, чиито елементи са от опорното множество \(X\).
\end{definition}

Процеса на изграждането на някой от тези вектори е следния: 
\begin{itemize}
    \item за първата позиция можем да изберем всеки елемент от \(X\), т.е. имаме \(n\) възможности
    \item за втората позиция имаме \(n - 1\) възможности, защото елементът от \(X\), избран за първа позиция,
    вече не може да се ползва
    \item аналогично за третата позиция има само \(n - 2\) възможности
    \item и т.н., за \(m\)-тата позиция има само \(n - (m - 1)\) възможности
\end{itemize}

Тогава 
\begin{align*}
    |K_\text{н}(n, m)| &= n * (n - 1) * (n - 2) * ... * (n - (m - 1)) \\
                       &= n * (n - 1) * ... * (n - m + 1) \\
                       &= \prod_{k = 0}^{m - 1} (n - k) = n^{\underline{m}\footnotemark}
\end{align*}
\footnotetext[1]{\(n^{\underline{m}}\) - "\(n\) на падаща степен \(m\)"} 

Този резултат не се получава от принципа на умножението.
Резултатът остава в сила и при \(m > n\), тогава дясната страна ще е 0.

\begin{example}
    Нека \mexpr{X = \{a, b, c\}, m = 2} \\
    Тогава \mexpr{K_\text{н}(3, 2) = \{(a, b); (a, c); (b, a); (b, c); (c, a); (c, b)} и 
    \mexpr{|K_\text{н}(3, 2)| = 3 * ... * (3 - 2 + 1) = 3 * 2 = 6}. \\
    Тук \mexpr{K_\text{н}(3, 2)} не е Декартово произведение нито на 3-елементно и 2-елементно множество, нито
    на 6-елементно и 1-елементно множество. Както на първа, така и на втора позиция се срещат и трите 
    елемента на \(X\).
\end{example}

\begin{example}
    Нека \mexpr{X = \{a, b, c\}, m = 4} \\
    Тогава \mexpr{K_\text{н}(3, 2) = \emptyset}, тъй като е невъзможно да не повторим елемент от \(X\) 
    съгласно принципа на Дирихле. \\
    Следователно \mexpr{|K_\text{н}(3, 4)| = 3^{\underline{4}} = 3 * 2 * 1 * 0 = 0}.
\end{example}

Частен случай е \(n = m\), тогава векторите се наричат пермутации.
Пермутациите на \(n\) на брой, два по два различни обекта, са разполаганията в линейна наредба на тези обекти.
Така \mexpr{|K_\text{н}(n, m)| = |K_\text{н}(n, n)| = n!}

\begin{definition}[Конфигурации без наредба и без повтаряне]
    Множеството от конфигурациите без наредба и без повтаряне с големина \(m\) 
    над опорно множество \(X\) с мощност \(n\) означаваме с "\(K(n, m)\)". Елементите му са 
    наредените \(m\)-елементните подмножества на опорното множество. Наричат се още комбинации.
\end{definition}

Въвеждаме релация \mexpr{R \subseteq K_\text{н}(n, m) \times K_\text{н}(n, m)} така:
\mexpr{\forall X, Y \in K_\text{н}(n, m): XRY \iff X} и \(Y\) имат едни и същи елементи.

\(R\) е релация на еквивалентност и всеки нейн клас на еквивалентност има мощност \(m!\), а 
\mexpr{|K(n, m)|} е броят на класовете на еквивалентност, като
\mexpr{|K(n, m)| = \frac{K_\text{н}(n, m)}{m!} = \frac{n * (n - 1) * (n - 2) * ... * (n - m + 1)}{m!} = 
\frac{n!}{m! * (n - m)!}}, което се нарича биномен коефициент.

\begin{example}
    Колко са възможните фишово при тото 6/49? \\
    Отговор: \mexpr{\frac{49 * 48 * 47 * 46 * 45 * 44}{6!} = \frac{10 068 347 520}{720} = 13 983 816} \\
    Причината да делим на \(6! = 720\) е, че няма значение в какъв ред се изтеглят числата.
\end{example}

\begin{definition}[Конфигурации без наредба с повтаряне]
    Множеството от конфигурациите без наредба с повтаряне с големина \(m\) 
    над опорно множество \(X\) с мощност \(n\) означаваме с "\(K_\text{п}(n, m)\)". Елементите му са 
    мултимножества с големина \(m\) на опорното множество.
    \mexpr{|K_\text{п}(n, m)| = \binom{n - 1 + m}{n - 1} = \binom{n - 1 + m}{m}}
\end{definition}

\begin{example}
    Нека \mexpr{X = \{a, b, c\}, m = 5}. Тогава
    \begin{align*}
        K_\text{п}(3, 5) = \{&\{a, a, a, a, a||\}, \\
                            &..., \\
                            &\{|b, b, b, b, b|\}, \\
                            &\{a, a, a, a, ||c\}, \\
                            &..., \\
                            &\{a, |b, b|, c, c\}, \\
                            &\{||c, c, c, c, c\}\}
    \end{align*}
    \mexpr{|K_\text{п}(3, 5)| = \binom{3 - 1 + 5}{3 - 1} = \binom{7}{2} = 7 * 3 = 21}
\end{example}

\begin{example}
    12 еднакви билета се раздават на 10 човека. По колко начина може да стане това? \\
    Р-е: множеството от хората е опорното множество. Броят на билетите е големината на конфигурацията, т.е.
    \(n = 10, m = 12\). \\
    Отговорът е: \mexpr{\binom{12 - 1 + 10}{12} = \binom{12 - 1 + 10}{10 - 1} = 239 930} \\
    Едно от раздаванията може да бъде: \(**||||***||*|*****|*|\), където първият човек е с 2 билета, 
    2-ят, 3-ят и 4-ят са с по 0 билета, 5-ят е с 3, 6-ят е с 0, 7-ят е с 1, 8-ят е с 5, 9-ят е с 1 и 
    10-ят е без билети.
\end{example}

\begin{example}
    Колко са фишовете в 6/49, ако след изтегляне на топка, тя бива връщана отново в сферата? \\
    Отговор: \mexpr{\binom{49 - 1 + 6}{6} = 25 827 165}
\end{example}

\subsection*{Биномен коефициент. Основни свойства}
Дефинираме е \mexpr{\binom{n}{m} = \frac{n!}{m! * (n - m)!} = \frac{n * (n - 1) * ... * (n - m + 1)}{m!}} 
за \mexpr{n, m \in \mathbb{N}}, което се нарича биномен коефициент.

Основни свойства:
\begin{itemize}
    \item ако \mexpr{m > n}, то \mexpr{\binom{n}{m} = 0}, което има комбинаторен смисъл на "има 0 начини 
    да изберем \(m\)-елементно подмножество на \(X\), ако \(|X| = n\) и \(m > n\)"
    \item \mexpr{\binom{n}{0} = \binom{n}{n} = 1}, което има комбинаторен смисъл на "има точно 1 начин да 
    изберем 0 нещо от \(n\) - не избираме нищо, т.е.избираме \(\emptyset\)" и "има точно 1 начин да 
    изберем \(n\) неща от \(n\) - избираме всичко, т.е. \(X\)"
    \item \mexpr{\binom{n}{1} = \binom{n}{n - 1} = n}, което има комбинаторен смисъл на "има \(n\) начина 
    да изберем 1 нещо от \(n\)" и "има \(n\) начина да изберем \(n - 1\) неща от \(n\)"
    \item изобщо, при \(m \le n\), е в сила \mexpr{\binom{n}{m} = \binom{n}{n - m}}
    \item при фиксиран горен индекс \(n\), сумата от всички биномни коефициенти е \(2^n\): \\
    \mexpr{\sum_{k = 0}^{n} \binom{n}{k} = \underbrace{\binom{n}{0}}_\text{1} + \underbrace{\binom{n}{1}}_\text{n} + 
    \underbrace{\binom{n}{2}}_\text{\(\frac{n * (n - 1)}{2}\)} + ... 
    + \underbrace{\binom{n}{n - 2}}_\text{\(\frac{n * (n - 1)}{2}\)} 
    + \underbrace{\binom{n}{n - 1}}_\text{n} + \underbrace{\binom{n}{n}}_\text{1} = 2^n}, което има 
    комбинаторен смисъл на "дясната страна брои всички подмножества на \(n\)-елементно множество - те са 
    \(2^n\), а лявата страна брои разбиването на всички подмножества по мощности". Това се извежда с броене 
    на характеристичните вектори с дължина \(n\), които са \mexpr{|\{0, 1\}^n| = |\{0, 1\}|^n = 2^n}.
    \item ако \(n, m \ge 1\) е в сила \mexpr{\binom{n}{m} = \binom{n - 1}{m} + \binom{n - 1}{m - 1}} - 
    триъгърник на Паскал
\end{itemize}

\subsection*{Теорема на Нютон}
\begin{theorem}
    \mexpr{\forall x, y \in \mathbb{R} \forall n \in \mathbb{N}: (x + y)^n = \sum_{k = 0}^{n} \binom{n}{k} * x^k * y^{n - k}}
\end{theorem}
\begin{proof}
    Лявата страна е \mexpr{\underbrace{(x + y) * (x + y) * ... * (x + y) * (x + y)}_\text{\(n\) множителя}}
    След отварянето на скобите, ще се получи сума от \(2^n\) събираеми от вида \(x^k * y^{n - k}\) по 
    всички \mexpr{k \in \{0, ..., n\}}. Разбиваме множеството от събираемите в \(n + 1\) множества по 
    степента на \(x\)(която диктува степента на \(y\)): \mexpr{x^0 * y^n, x^1 * y^{n - 1}, ..., x^n * y^0}. 
    Коефициентът пред \(x^k * y^{n - k}\) е броят на появите това събираемо, като съобразяваме, че
    \(x^k\) "идва" от \(k\) на брой множителя (останалите множители дават \(y^{n - k}\)). Тези \(k\) 
    множителя можем да изберем от всички \(n\) множителя по \mexpr{\binom{n}{k}} начина.
    Примерно, \(x^n * y^0\) се появява само веднъж, понеже за него трябва да "дойде" \(x\) от всеки 
    множител, \(x^{n- 1} * y^1\) се появява точно \(n\) пъти, защото от един множител "идва" \(y\), а от 
    останалите - \(x\), като този един множител може да изберем по \(n\) начина и т.н.
\end{proof}

Обобщение:
\mexpr{(x_1 + ... + x_k)^n = \sum_{n_1 + ... + n_k = n} \binom{n!}{n_1!n_2!...n_k!} x_1^{n_1} ... x_k^{n_k}}
Изразът \mexpr{\binom{n!}{n_1! ... n_k!}}, където \mexpr{n_1 + ... + n_k = n}, има същата стойност като 
\mexpr{\binom{n}{n_1} * \binom{n - n_1}{n_2} * ... * \binom{n - n_1 - ... - n_{k - 1}}{n_k}} и се нарича 
мултиномен коефициент. Той брои пермутации с повторения (на мултимножество).

\subsection*{Доказателства на комбинаторни тъждества с комбинаторни разсъждения}
\begin{example}
    Ще докажем, че \mexpr{\binom{2n}{n} = \sum_{k = 0}^n \binom{n}{k}^2}. \\
    \mexpr{\sum_{k = 0}^n \binom{n}{k}^2 = \sum_{k = 0}^n \binom{n}{k}\binom{n}{n - k}} \\
    % TODO: circles
    Задачата се свежда до "по колко начина можем да сложим \(p\) единици и \(q\) нули в редица?". \\
    Общо \(p + q\) булеви цифри. Това са характеристичните вектори с дължина \(p + q\) и точно \(p\) единици.
    Те съответсват биективно на \(p\)-елементните подмножества на \(p + q\)-елементно множество. Тези 
    подмножества са \mexpr{\binom{p + q}{p}}. Тогава и въпросните характеристичните вектори са толкова.
\end{example}