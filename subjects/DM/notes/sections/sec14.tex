\section{Тема 14}

Алгоритмична схема за обхождане на графи. \\
Вход: ориентиран мултиграф \(G\) \\
Променливи: м-во от върхове \(S\), връх \(x\) и булев масив \(visited[1, ..., n]\) \\
Инициализирай \(visited\) с \(FALSE\): \mexpr{S \leftarrow \{1\}, visited[1] \leftarrow TRUE}
\begin{enumerate}
    \item Ако \mexpr{S = \emptyset}, прекрати алгоритъма
    \item В противен случай, извади елемент от \(S\) и го сложи в \(x\)
    \item За всяко ребро \((x, y)\), ако \mexpr{visited[y] = FALSE}:
    \begin{enumerate}
        \item то слагаме \(y\) в \(S\) и правим \mexpr{visited[y] \leftarrow TRUE}
        \item иначе, прескачаме \(y\)
    \end{enumerate}
    \item Отиваме на 1)
\end{enumerate}

\subsection*{Обхождания на графи – в дълбочина и ширина. Дърво на обхождането}
\bu{Breadth-First Search (BFS)} е алгоритъм за обхождане на графи в ширина. Изграден е от алгоритмичната 
схема за обхождане на графи, като м-вото \(S\) е реализирано чрез абстрактен тип данни опашка (FIFO). 
Алгоритъвът започва от даден стартов връх, обхожда неговите съседи, като при това обхожда ребрата, с 
които ги достига, после обхожда съседите на съседите (различни от стартовия връх), и т.н., обхождайки 
върховете по нарастване на разстоянията от стартовия връх.

BFS изгражда така нареченото \underline{дърво на обхождане}. Ако графът е неориентиран, то това е 
покриващо дърво за свързаната компонента, съдържаща стартовия връх, което е кореново дърво с корен 
стартовия връх, а ребро \(e\) от графа влиза в дървото \totw BFS открива непосетен връх чрез \(e\). 
С други думи, дървото на обхождане показва как BFS е откривал непосетени върхове. Това дърво касае само 
обхождането на върховете. Всяко ребро бива обхождано от BFS, но не всяко ребро влиза в дървото на 
обхождане.

\bu{Depth-First Search (DFS)} е алгоритъм за обхождане за на графи (и по-специално дървета) в дълбочина. 
При него се избира даден връх, който се обозначава като корен (връх без предшественици) и обхождането 
започва от него. Последователно се посещават всички следващи върхове до достигането на връх без 
наследници (листо), след което се осъществява \underline{търсене с връщане назад} до достигане на нова 
крайна точка или цялостно реализирано обхождане - към корена.

\subsection*{Ойлерови/Хамилтонови обхождания/графи}

\begin{definition}
    Нека \(G\) е неориентиран свързан мултиграф. \\
    \underline{Ойлеров цикъл} в \(G\) е цикъл (не непременно прост), който съдържа всяко ребро точно веднъж. \\
    \underline{Ойлеров път} е в \(G\) е път (не непременно прост), който съдържа всяко ребро точно веднъж. \\
    Граф \(G\) е \underline{Ойлеров}, ако има Ойлеров цикъл.
\end{definition}

\begin{definition}
    Нека е даден граф \(G\). \\
    \underline{Хамилтонов цикъл} в \(G\) е всеки цикъл в \(G\), който съдържа всички върхове на \(G\). \\
    \underline{Хамилтонов път} в \(G\) е всеки път в \(G\), който съдържа всички върхове на \(G\). \\
    Граф \(G\) е \underline{Хамилтонов}, ако има Хамилтонов цикъл.
\end{definition}

\begin{note}
    Няма как в един граф едновременно да има Ойлеров път и Ойлеров цикъл. \\
    Ойлеров път (НДУ: графът да е свързан и всички върхове да имат четна степен с изключение на два от тях) 
    \\ \(\not = \) \\
    Ойлеров цикъл (НДУ: графът да е свързан и всички върхове да са с четна степен)
\end{note}

\subsection*{Теореми за съществуване на Ойлеров цикъл и Ойлеров път в неориентиран и ориентиран мултиграф}

\begin{theorem}[за съществуване на Ойлеров цикъл в неориентиран свързан мултиграф]
    Неориентиран свързан мултиграф \(G\) има Ойлеров цикъл \totw всеки връх има четна степен.
\end{theorem}

\begin{proof}
    \(\newline\Rightarrow)\) Да допуснем, че \(G\) има Ойлеров цикъл. \\
    Да разгледаме произволен връх \mexpr{w \in V(G)}. Тогава \mexpr{w \in V(c)}, където \(c\) е 
    Ойлеров цикъл в \(G\).
    \begin{enumerate}
        \item Ако \(w\) няма примки, то няма как два съседни върха в \(c\) да са \(w\). Следователно, на 
        всяка поява на \(w\) в \(c\) съответстват точно 2 ребра - съседните елементи на \(w\) в цикъла - 
        като за всеки 2 различни появи на \(w\), двете бройки ребра нямат общ елемент. Следователно, 
        множеството от всички тези двойки ребра, върху всички появи на \(w\) в \(c\), е точно \(J(w)\). 
        Нека \(w\) се появява точно \(t\) пъти в \(c\). Тогава \mexpr{|J(w)| = 2.t}. От \mexpr{d(w) = |J(w)|} 
        следва, че степента на \(w\) е четно число.
        \item Нека \(w\) има \(q\) примки \(e_1, ..., e_q\) и \(w\) се поява точно \(t\) пъти в \(c\). 
        Тогава \mexpr{t \ge q}. При наличието на примки на \(w\) има съседни появи на \(w\) в \(c\), тоест 
        в \(c\) има точно \(q\) подредици \mexpr{we_iw, 1 \le i \le q}, като някои от тези подредици може 
        да имат общ край \(w\). Всяка такава триелеминтна подредица, отговаряща на дадена примка, има принос 
        +2 към степента на \(w\). Максималните по включване подредици са от вида \mexpr{we_{j_1}, ..., e_{j_r}w}, 
        където \mexpr{e_{j_1}, ..., e_{j_r} \in \{e_1, ..., e_q\}}, са точно \(t - q\) на брой. Нека \(E'\) e 
        м-вото от всички ребра, инцидентни с \(w\), които не са примки. Тъй като теци максимални по 
        включване подредици никога не са съседни, тоест между всеки две от тях в \(c\) има поне един връх, който 
        не е \(w\), като \mexpr{|E'| = 2.(t - q)}. Но \mexpr{d(w) = |E'| + 2.q = 2.(t - q) + 2.q = 2.t}, което 
        е четно число.
    \end{enumerate}
    \(\Leftarrow)\) Да допуснем, че всеки връх в \(G\) е от четна степен. \\
    Ще докажем, че съществува Ойлеров цикъл конструктивно чрез следния алгоритъм (на Hierholtzer): \\
    Вход: свързан мултиграф \(G\) с поне едно ребро \\
    Изход: Ойлеров цикъл \(c\) в \(G\) \\
    Променливи: \(w\) и \(x\) - върхове, \(s\) - цикъл \\
    \begin{enumerate}
        \item \(c \leftarrow \epsilon\) (празната редица). Маркирай всички ребра на \(G\) 
        като неизползвани
        \item Избери произволен връх \(a\), инцидентен с неизползвано ребро
        \item Присвои \mexpr{x \leftarrow a, w \leftarrow a, s \leftarrow a}
        \item Ако \(w\) има поне едно инцидентно неизползвано ребро \mexpr{e = (w, v)}:
        \begin{enumerate}
            \item присвои \mexpr{s \leftarrow s, e, v} и маркирай \(e\) като използвано
            \item присвои \(w \leftarrow v\) и иди на 4)
        \end{enumerate}
        \item В противен случай, вмъкни \(s\) в \(c\), тоест:
        \begin{itemize}
            \item ако \(c\) е празен, то присвои \(c \leftarrow s\)
            \item в противен случай \(c\) съдържа поне една поява на върха \(y\), който е начало и 
            край на \(s\). Замени коя да е поява на \(y\) в \(c\) с редицата \(s\)
        \end{itemize}
        \item Ако няма неизползвани ребра, върни \(c\) и прекрати алгоритъма
        \item В противен случай, избери произволен връх \(b\) от \(c\), инцидентен с неизползвано ребро
        \item Присвои \mexpr{x \leftarrow b, w \leftarrow b, s \leftarrow b} и иди на 4)
    \end{enumerate} 
\end{proof}

\begin{note}
    Идеята на алгоритъма е следната - в началото е неизползвано, а \(c\) е празен (празната редица). 
    Използваме временен цикъл \(s\) и текущ връх \(w\). Започвайки от произволен връх \(x\), койте е 
    инцидентен с поне едно неизползвано ребро, инициализираме \mexpr{w \leftarrow x, s \leftarrow x}. 
    После изпълняваме, докато е възможно, следното: избираме произволно неизползвано ребро \(e \in J(w)\), 
    маркираме \(e\) като използвано и "преминаваме" през \(e\), тоест, ако другият край на \(e\) е \(v\), 
    добавяме \(e\) и \(v\) към \(s\) като \mexpr{s \leftarrow s, e, v}, текущият връх \(w\) става \(v\), 
    тоест правим \(w \leftarrow v\). Правим това, докато можем, а не докато стигнем до текущ връх \(w\), 
    който няма инцидентно необходено ребро. Тъй като ребрата са краен брой и ние променяме статусите на 
    ребрата от неизползвани в използвани, то рано или късно ще се окажем във връх \(w\), който няма 
    неизползвани инцидентни ребра. В този момент \(w\)е точно този връх \(x\), от който започнахме 
    обхождането. В този момент \(s\) е цикъл. Вмъкваме \(s\) в \(c\). Ако всички ребра са използвани, 
    връщаме \(c\) и терминираме алгоритъма. В противен случай продължаваме по следния начин - тъй като 
    \(G\) е свързан, в \(c\) задължително има връх \(b\), който е инцидентен с поне едно необходено ребро. 
    Тогава присвояваме \mexpr{w \leftarrow b, s \leftarrow b} и отново изпълняваме итерираното добавяне 
    на ребра и върхове към \(s\), докато можем.
\end{note}

\begin{theorem}[за съществуване на Ойлеров път в неориентиран свързан мултиграф]
    Свързан неориентиран мултиграф \(G\) има Ойлеров път, който не е цикъл, \totw точно 2 върха са от 
    нечетна степен.
\end{theorem}

\begin{proof}
    \(\newline\Rightarrow)\) Нека \(G\) има път \(p\), който съдържа всяко ребро точно веднъж и има 
    различни краища \(u\) и \(v\). Ще покажем, че \(d(u)\) и \(d(v)\) са нечетни, а всички останали върхове 
    имат четни степени. \\
    Имаме \(p = u, ..., v\). Добавяме едно ново ребро \(e\) между \(u\) и \(v\), получавайки мултиграф 
    \(G'\). Тогава \(G'\) има Ойлеров цикъл \(c\), състоящ се от \(p\) и новото ребро \(e\), т.е. 
    \mexpr{c = p, e, u}. Тогава всички върхове в \(G'\) са от четна степен. Имаме, че 
    \mexpr{d_{G'}(u) = d_G(u) + 1, d_{G'} (v) = d_G (v) + 1} и \mexpr{\forall x \in V(G) \setminus 
    \{u, v\}: d_{G'} (x) = d_G (x)}, откъдето следва, че \(d_G(u)\) и \(d_G(v)\) са нечетни, а 
    останалите върхове в \(G\) са с четни степени.
    \(\newline\Leftarrow)\) Нека в \(G\) има точно два върха от нечетна степен. Ще покажем, че в \(G\) 
    има Ойлеров \(u-v\) път. \\
    Добавяме едно ново ребро \(e\) между \(u\) и \(v\), получавайки мултиграф \(G''\). Тогава в \(G''\) 
    всички върхове са от четна степен, откъдето следва, че \(G''\) има Ойлеров цикъл \(c\). Изтриваме 
    \(e\) от \(c\) и получаваме Ойлеров \(u-v\) път.
\end{proof}

\begin{theorem}[за съществуване на Ойлеров цикъл в ориентиран свързан мултиграф]
    Краен ориентиран свързан мултиграф \(G = (V, E, f_G)\) е Ойлеров \totw за всеки връх полустепента на 
    входа и изхода съвпадат.
\end{theorem}

\begin{proof}
    (???) Нека ребрата на \(G\) образуват Ойлеров цикъл. Тогава за всеки връх полустепента на входа и изхода 
    съвпадат, защото графът е Ойлеров, а ако не съвпадат, то няма да могат да се обходят всички ребра.
\end{proof}

\begin{theorem}[за съществуване на Ойлеров път в ориентиран свързан мултиграф]
    Краен ориентиран свързан мултиграф \(G = (V, E, f_G)\) съдържа Ойлеров път \totw само за два от 
    върховете му полустепените на входа и изхода не съвпадат, като в единия връх полустепента на изхода е 
    с единица по-голяма от полустепента на входа, а при другия обратно.
\end{theorem}

\begin{proof}
    Нека ребрата на \(G\) образуват Ойлеров път от \(v_i\) до \(v_j\). Добавяме 
    реброто \(e \not \in E\) и додефинираме \mexpr{f_{G'}(e) = (v_i, v_j)}. Пътят се превръща в Ойлеров 
    цикъл за новополучения мултиграф \(G'\) и следователно в него всички върхове имат еднаква полустепен 
    на входа и изхода. Добавянето на реброто \((v_i, v_j)\) е увеличило с 1 само полустепента на изхода 
    на \(v_j\) и полустепента на входа на \(v_i\). Следователно в \(G\) всички върхове имат еднаква 
    полустепен на входа и изхода, освен два от върховете, за които единия има с единица по-голяма 
    полустепен на входа отколкото на изхода, а при другия обратно.
\end{proof}

\subsection*{Бележки}
