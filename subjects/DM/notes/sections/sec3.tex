\section{Тема 3}

\subsection{Индуктивно дефенирани множества}
\boxt{Аксиома за индукцията} \\
За тази аксиома е удачно да мислим за нея като за конструкция или безкрайна процедура, която генерира
множество, стартирайки от някаква база и прилагайки итеративно някакви операции.

Нека е дадена непразно множество \(M_0\), което наричаме \\ \bu{базово множество}, и непразно множество
от операции \(F\), приложими в тази конструкция.

\begin{itemize}
    \item Включваме елементите на \(M_0\) в \(M\), т.е. \mexpr{M \leftarrow M_0}.
    \item Прилагаме неограничено следното:
        \begin{itemize}
            \item[-] нека \(M^{'}\) е множеството от елемнитите, които се получават при всевъзможните 
            прилагания на операциите от \(F\) върху текущото \(M\)
            \item[-] добавяме \(M^{'}\) към \(M\), т.е. \(M \leftarrow M \cup M^{'}\)
        \end{itemize}
\end{itemize}

Така полученото \(M\) е множество. Пишем \mexpr{M = (M_0, F)}.

Множества, генерирани чрез безкрайната процедура от аксиомата за индукцията, наричаме индуктивно генерирани
множества.

\bu{Пример:} множеството на естествените числа \(\mathbb{N}\) \\
При него \mexpr{M_0 = \{0\}}, а \(F\) съдържа една единствена операция - добавяне на единица.

\subsection{Доказателства по индукция - обикновена, силна, структурна. Примери}
Нека е даден предикат \(P(n)\) и трябва да докажем \mexpr{\forall x \in \mathbb{N} : P(n)}.

\boxt{Обикновена индукция} \\
Схемата на доказателствата по индукция върху естествените числа е следната:
\begin{enumerate}
    \item[(База)] доказваме \(P(0)\), като просто проверяваме истиността на предиката за \(n = 0\)
    \item[(И.П.)] допускаме \(P(n)\) за произволно \mexpr{n \in \mathbb{N}} и въз основа на това 
    допускане доказваме \(P(n + 1)\) (И.С.)
\end{enumerate}

\bu{Пример:}

\boxt{Силна индукция} \\
Схемата на доказателствата със силна индукция върху естествените числа е следната:
\begin{enumerate}
    \item[(База)] доказваме \(P(0)\), като просто проверяваме истиността на предиката за \(n = 0\)
    \item[(И.П.)] допускаме, че за произволно \mexpr{n \in \mathbb{N}} са изпълнени \mexpr{P(0), P(1), ..., P(n)}
    \item[(И.С.)] въз основа на тези допускания доказваме \(P(n + 1)\) 
\end{enumerate}

\bu{Пример:}

\boxt{Структурна индукция} \\
В по-общия случай доказваме предикат \(P(x)\), където домейнът е някакво индуктивно дефенирано 
множество \mexpr{M = (M_0, F)}. Схемата на доказателство е следната:
\begin{itemize}
    \item[(База)] за всеки елемент \(x\) от \(M_0\) проверяваме истиността на \(P(x)\).
    \item[(И.П.)] допускаме \(P(x)\) за произволно \mexpr{x \in M}
    \item[(И.С.)] въз основа на това допускане доказваме, че за всеки елемент \(y\), който се 
    получава при прилагането на операциите от \(F\) върху текущото \(M\), то \(P(y)\) е вярно
\end{itemize}

\bu{Пример:}

\subsection{Наредена двойка. Декартово произведение}
(Дефиниция на Kuratowski) Всяко множество \mexpr{\{\{a\}, \{a, b\}\}} наричаме \\ \bu{наредена двойка} с 
първи елемент \(a\) и втори елемент \(b\). Бележим "\mexpr{(a, b)}". \\

Нека \(A\) и \(B\) са множества. \bu{Декартовото произведение} на \(A\) и \(B\) е множеството 
\mexpr{A \times B = \{(a, b) | a \in A \land b \in B\}}.
