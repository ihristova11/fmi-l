\section{Тема 8}

% Принципи на изброителната комбинаторика: принцип на Дирихле, принцип
% на биекцията, принципи на събирането (разбиването) и изваждането,
% принцип на умножението (Декартовото произведение) и делението. 
\subsection{Принципи на изброителната комбинаторика}

\begin{principle}[на Дирихле]
    Ако \(X\) и \(Y\) са крайни множества и \mexpr{|X| > |Y|}, то не съществува инекция \mexpr{f: X \to Y}. \\ \\
    \underline{Алтернативна формулировка}: ако има \(m\) ябълки в \(n\) чекмеджета и \(m > n\), то в поне едно 
    чекмедже има повече от една ябълка. \\
    \underline{Обобщение}: ако има \mexpr{k * n + 1} ябълки в \(n\) чекмеджета, то в поне едно чекмедже има повече 
    от \(k\) ябълки. 
\end{principle}

\begin{principle}[на биекцията]
    Нека \(X\) и \(Y\) са крайни множества. Тогава \mexpr{|X| = |Y|} \totw съществува биекция \mexpr{f: X \to Y}.
\end{principle}

\begin{principle}[на събирането (разбиването)]
    Нека е дадено множество \(X\) и разбиване \mexpr{Y = \{Y_1, ..., Y_k\}} на \(X\). Тогава 
    \mexpr{|X| = |Y_1| + ... + |Y_k|}. \\

    Това остава в сила дори някои от множествата \mexpr{Y_1, ... Y_k} да са празни, защото мощностите на 
    празните \(Y_i\) са нули и те не се отразяват на сумата.
\end{principle}

\begin{principle}[на изваждането]
    Нека е дадено множество \(X\) в универсум \(U\). Тогава \mexpr{|X| = |U| - |\overline{X}|}.

    Очевидно \mexpr{\{X, \overline{X}\}} е разбиване на универсума, така че от принципа на събирането 
    имаме, че \mexpr{|U| = |X| + |\overline{X}|}.
    Не е възможно \(\overline{X}\) да е празно, но дори тогава твърдението остава в сила.
\end{principle}

\begin{principle}[на умножението]
    Нека \(X\) и \(Y\) са множества. Тогава \mexpr{|X \times Y| = |X| * |Y|.}
\end{principle}

\begin{principle}[на делението]
    Нека \(X\) е множество и \(R \subseteq X^2\) е релация на еквивалентност. Нека \(X\) има \(k\) класа 
    на еквивалентност и всеки клас на еквивалентност има мощност \(m\). Тогава 
    \mexpr{m = \frac{|X|}{k}}.
\end{principle}

\subsection{Принцип на включването и изключването}
\begin{principle}[на включването и изключването]
    Нека е дадено покриване на множество и търсим мощността на множеството, като събираме и изваждаме 
    мощностите на дяловете на покриването, техните сечения по двойки, по тройки и т.н.

    \underline{Обща формулировка}: \\
    За всяко \(n \ge 1\), за всеки \(n\) множества \mexpr{A_1, A_2, ..., A_n}: \\
    \begin{equation}
        |A_1 \cup ... \cup A_{n - 1}| = \sum_{1 \le i \le n - 1} |A_i| - \sum_{1 \le i < j \le n - 1} |A_i \cap A_j| 
        + ... + (-1)^{n - 2}|A_1 \cap ... \cap A_{n - 1}|
    \end{equation}
\end{principle}

% TODO: align formulas
\begin{proof}[Доказателство с индукция по \(n\):]
    $ $\newline
    \bu{База}: \mexpr{n = 1} \\
    Тогава (1) става \mexpr{|A_1| = |A_1|}. \\

    \bu{Индукционно предположение}: нека твърдението е изпълнено за всеки \(n - 1\) множества, т.е.
    \begin{equation}
        |A_1 \cup ... \cup A_{n - 1}| = \sum_{1 \le i \le n - 1} |A_i| - \sum_{1 \le i < j \le n - 1} |A_i \cap A_j| 
        + ... + (-1)^{n - 2}|A_1 \cap ... \cap A_{n - 1}|
    \end{equation}

    \begin{align*}
        |A_1 \cup ... \cup A_{n - 1}| &= \sum_{1 \le i \le n - 1} |A_i| - \sum_{1 \le i < j \le n - 1} |A_i \cap A_j| 
        + ... + (-1)^{n - 2}|A_1 \cap ... \cap A_{n - 1}|
    \end{align*}

    \bu{Индукционна стъпка}: ще докажем твърдението за всеки \(n\) множества. \\
%    \begin{align*}
%        |A_1 \cup ... \cup A_{n - 1} \cup A_n| = |\underbrace{(A_1 \cup ... \cup A_{n - 1})}_\text{X} \cup 
%        \underbrace{A_n}_\text{Y} = \\ |\underbrace{(A_1 \cup ... \cup A_{n - 1})}_\text{X}| + |\underbrace{A_n}_\text{Y}| 
%        - |\underbrace{(A_1 \cup ... \cup A_{n - 1})}_\text{X} \cap \underbrace{A_n}_\text{Y}|
%    \end{align*}

\end{proof}